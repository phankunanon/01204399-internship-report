% !TEX program = xelatex

\documentclass[16pt,a4]{internshipreport}

\usepackage{lipsum}  

\title{การเขียนรายงานฝึกงานที่เขียนเทมเพลทนานกว่ารายงาน}
\entitle{Writing a nonsensible internship report just to test the template}
\location{ครัวป้าวง ป่ายุบใน}
\date{2562}
\author{ศิระกร ลำใย}
\studentid{5910500023}

\begin{document}

\maketitle

\section*{บทคัดย่อ}
\lipsum[2-4]

\section*{กิตติกรรมประกาศ}
\lipsum[2-4]

\tableofcontents

\section{บทนำ}

\subsection{ความสำคัญและที่มา}
หลักสูตรวิศวกรรมศาสตรบัณฑิต สาขาวิชาวิศวกรรมคอมพิวเตอร์ หลักสูตรปรับปรุง พ.ศ 2556
ระบุให้ผู้เรียนทุกคนต้องเข้ารับการฝึกงาน เพื่อเพิ่มพูนประสบการณ์ในการเรียนรู้ที่ไม่อาจหาได้ในห้องเรียน
และเป็นการฝึกทักษะของวิศวกรในการทำงานจริง

คณะวิศวกรรมคอมพิวเตอร์ และภาควิชาวิศวกรรมคอมพิวเตอร์ จึงกำหนดให้มีการเรียนการสอนในรายวิชา
01204399 หรือการฝึกงาน แบ่งเป็นการฝึกงานภาคฤดูร้อนสำหรับนิสิตที่ไม่ได้สหกิจ
และฝึกงานต่อเนื่องในช่วงเวลาของภาคฤดูร้อนและภาคการศึกษาต้นของมหาวิทยาลัยสำหรับนิสิตที่สหกิจ
จึงเป็นที่มาของรายงานเล่มนี้ซึ่งเป็นหนึ่งในข้อกำหนด/ข้อบังคับของการฝึกงาน

\subsection{วัตถุประสงค์การปฏิบัติงาน}
\begin{itemize}
    \item เพิ่มพูนประสบการณ์ในการเรียนรู้ที่ไม่อาจหาได้ในห้องเรียน
    \item เพื่อพัฒนาทักษะการทำงาน การสื่อสาร และทักษะ soft skills อื่นๆ
    \item เพื่อเป็นการเตรียมตัวในการทำโครงงานวิศวกรรมคอมพิวเตอร์
\end{itemize}

\subsection{ขอบเขต}
ไว้มาเขียน

\subsection{ประวัติและรายละเอียดสถานประกอบการ}
\textbf{สถาบันวิทยสิริเมธี (VISTEC)} เป็นบัณฑิตวิทยาลัย (graduate school) ซึ่งมุ่งเน้นความเป็นเลิศในการทำวิจัย 
ตั้งอยู่ในพื้นที่วังจันทร์วัลเลย์ (Wangchan Valley) และเขตนวัตกรรมระเบียงเศรษฐกิจพิเศษภาคตะวันออก
(Eastern Economic Corridor of Innovation: EECi) เลขที่ 555 หมู่ 1 ตำบลป่ายุบใน อำเภอวังจันทร์ จังหวัดระยอง
ก่อตั้งขึ้นเมื่อปี พ.ศ. 2558 โดยมูลนิธิพลังสร้างสรรค์นวัตกรรม ภายใต้การสนับสนุนเงินทุนจากบริษัท
ในกลุ่มของการปิโตรเลียมแห่งประเทศไทย (ปตท.)

VISTEC มุ่งเน้นการจัดการศึกษาด้านวิทยาศาสตร์ วิศวกรรม และเทคโนโลยี โดยมีศูนย์วิจัยวิทยาศาสตร์และเทคโนโลยีชั้นแนวหน้า
(Frontier Research Center) ซึ่งเป็นศูนย์กลาง ในการเสริมสร้างความเข้มแข็งทางการวิจัย
และให้การสนับสนุนด้านทุนการวิจัยแก่สถาบันฯ เป็นศูนย์รวมนักวิจัยที่มีความเชี่ยวชาญสูง ช่วยขับเคลื่อนการดำเนินงานด้านการศึกษา 
ิจัย การสร้างนวัตกรรม สร้างความร่วมมือทางด้านวิจัยกับสถาบันการศึกษา ภาคธุรกิจ ภาคอุตสาหกรรม
และหน่วยงานด้านการวิจัยวิทยาศาสตร์และเทคโนโลยี

\textbf{ห้องปฏิบัติการเบรน} (Bio-inspired Robotics and Neural Engineering: BRAIN)
ณ สำนักวิชาวิทยาศาสตร์และเทคโนโลยีสารสนเทศ สถาบันวิทยสิริเมธี มุ่งเน้นศึกษาการสร้างหุ่นยนต์ที่มีลักษณะร่วมกับกายวิภาค (anatomy) ของสิ่งมีชีวิต และใช้เทคโนโลยีจำพวก Machine Learning หรือ Deep Learning ในการจำแนก วิเคราะห์
และประมวลผลคลื่นสมองของมนุษย์ เพื่อสร้างส่วนติดต่อผู้ใช้ผ่านสมอง (Brain Controlled Interfaces: BCIs)

ลักษณะงานที่ได้รับผิดชอบจากห้องปฏิบัติการฯ เป็นงานของผู้ช่วยนักวิจัย (Research Assistant: RA) ซึ่งช่วยนิสิตระดับ
บัณฑิตศึกษาในการเตรียมการทดลอง ออกแบบ และพัฒนาเครื่องมือวัดผล ควบคุมการทดลอง และทดสอบสมมติฐานเพื่อตีพิมพ์
องค์ความรู้ในวารสารวิชาการต่อไป

ที่ปรึกษาและผู้ควบคุมการฝึกงานในครั้งนี้ คืออ.ดร. ธีรวิทย์ วิไลประสิทธิ์พร หัวหน้าหน่วยวิจัย (Principal Investigator: PI)
และมีระยะเวลาปฏิบัติงานประมาณ 2 เดือน กล่าวคือตั้งแต่วันที่ 4 มิถุนายน ถึง 31 กรกฎาคม 2562
\end{document}